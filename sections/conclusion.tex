
\section{Conclusión}
\textbf{Mario Eduardo Sánchez Mejía:\\}
La práctica experimental permitió una comprensión profunda de los sistemas de control electrónico en aplicaciones reales, demostrando cómo los principios teóricos de los semiconductores se traducen en soluciones prácticas y funcionales. El manejo de señales digitales y analógicas a través de microcontroladores reveló la versatilidad de estos sistemas para adaptarse a diferentes necesidades de control y automatización. Además, la implementación del sistema de iluminación con transistores BJT demostró la importancia de comprender los principios fundamentales de la electrónica de potencia, incluyendo la polarización adecuada, el manejo de cargas inductivas y la importancia de la protección de componentes.

\textbf{Fidel Alberto Zarco Áviles:\\}
El desarrollo práctico evidenció que la selección adecuada de componentes electrónicos es crucial para el éxito de un sistema de control, destacando la importancia de comprender las características y limitaciones de cada dispositivo. La experiencia con diferentes tipos de transistores, especialmente con el MOSFET IRLZ14, permitió entender las ventajas y limitaciones de cada tecnología, así como la importancia de considerar factores como la disipación de calor, la eficiencia energética y la protección de los componentes en el diseño de circuitos electrónicos. La implementación del control PWM para la regulación de intensidad luminosa demostró la versatilidad de los sistemas digitales en aplicaciones analógicas, mientras que el uso de potenciómetros como interface de control evidenció la importancia de considerar la experiencia del usuario en el diseño de sistemas electrónicos.