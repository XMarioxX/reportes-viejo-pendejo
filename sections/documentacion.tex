\section{Documentacion del Piano Digital con Tema de Tetris}

\subsection{Descripcion General}
Este proyecto implementa un piano digital con Arduino que incluye ocho notas musicales basicas y una funcion especial que reproduce el tema musical de Tetris. El sistema utiliza botones como teclas de piano y proporciona retroalimentacion visual mediante un LED.

\begin{figure}[H]
	\centering
	\begin{lstlisting}[
		language=C++,
		basicstyle=\ttfamily\scriptsize,
		numbers=left,
		numberstyle=\tiny\color{gray},
		backgroundcolor=\color{codeBackground},
		commentstyle=\color{comentarios},
		keywordstyle=\color{identificador},
		stringstyle=\color{cadena},
		emph={pinMode,digitalWrite,tone,delay,HIGH,LOW},
		emphstyle={\color{blue}\bfseries},
		frame=single,
		frameround=tttt,
		rulecolor=\color{arduinoBlue},
		xleftmargin=5mm,
		xrightmargin=2mm,
		breaklines=false,
		columns=fullflexible,
		keepspaces=true,
		basewidth=0.5em 
]
int tonePin = 4;    
int ledMusic = 3;   
	\end{lstlisting}
	\caption{Definicion de Variables Globales}
	\label{fig:variables-globales}
\end{figure}

La figura \ref{fig:variables-globales} muestra la definicion de los pines principales del sistema. El \texttt{tonePin} se utiliza para la generacion de sonidos, mientras que \texttt{ledMusic} controla el LED indicador.

\subsection{Configuracion Inicial}
\begin{figure}[H]
	\centering
	\begin{lstlisting}[
		language=C++,
		basicstyle=\ttfamily\scriptsize,
		numbers=left,
		numberstyle=\tiny\color{gray},
		backgroundcolor=\color{codeBackground},
		commentstyle=\color{comentarios},
		keywordstyle=\color{identificador},
		stringstyle=\color{cadena},
		emph={pinMode,digitalWrite,tone,delay,HIGH,LOW},
		emphstyle={\color{blue}\bfseries},
		frame=single,
		frameround=tttt,
		rulecolor=\color{arduinoBlue},
		xleftmargin=5mm,
		xrightmargin=2mm,
		breaklines=false,
		columns=fullflexible,
		keepspaces=true,
		basewidth=0.5em 
]
void setup() {
	for (int pin = 5; pin <= 13; pin++) {
		pinMode(pin, INPUT);    
	}
	pinMode(tonePin, OUTPUT);   
	pinMode(ledMusic, OUTPUT);  
}
	\end{lstlisting}
	\caption{Funcion de Configuracion}
	\label{fig:setup-function}
\end{figure}

La funcion \texttt{setup()} mostrada en la figura \ref{fig:setup-function} realiza la configuracion inicial del sistema. Los pines 5-13 se configuran como entradas para los botones, el pin 4 (tonePin) como salida para el buzzer, y el pin 3 (ledMusic) como salida para el LED indicador.

\subsection{Bucle Principal y Control de Notas}
\begin{figure}[H]
	\centering
	\begin{lstlisting}[
		language=C++,
		basicstyle=\ttfamily\scriptsize,
		numbers=left,
		numberstyle=\tiny\color{gray},
		backgroundcolor=\color{codeBackground},
		commentstyle=\color{comentarios},
		keywordstyle=\color{identificador},
		stringstyle=\color{cadena},
		emph={pinMode,digitalWrite,tone,delay,HIGH,LOW},
		emphstyle={\color{blue}\bfseries},
		frame=single,
		frameround=tttt,
		rulecolor=\color{arduinoBlue},
		xleftmargin=5mm,
		xrightmargin=2mm,
		breaklines=false,
		columns=fullflexible,
		keepspaces=true,
		basewidth=0.5em 
]
void loop() {
	if (digitalRead(13) == HIGH) {
		tone(tonePin, 262, 100);  
	}
	if (digitalRead(12) == HIGH) {
		tone(tonePin, 294, 100);  
	}
	if (digitalRead(11) == HIGH) {
		tone(tonePin, 330, 100);  
	}
	if (digitalRead(10) == HIGH) {
		tone(tonePin, 349, 100);  
	}
	if (digitalRead(9) == HIGH) {
		tone(tonePin, 392, 100);  
	}
	if (digitalRead(8) == HIGH) {
		tone(tonePin, 440, 100);  
	}
	if (digitalRead(7) == HIGH) {
		tone(tonePin, 494, 100);  
	}
	if (digitalRead(6) == HIGH) {
		tone(tonePin, 523, 100);  
	}
	if (digitalRead(5) == HIGH) {
		digitalWrite(ledMusic, HIGH);
		tetrisTheme();
		digitalWrite(ledMusic, LOW);
	}
	delay(100);  
}
	\end{lstlisting}
	\caption{Bucle Principal y Control de Notas}
	\label{fig:loop-function}
\end{figure}

\subsection{Funcion Auxiliar de Control LED-Tono}
\begin{figure}[H]
	\centering
	\begin{lstlisting}[
		language=C++,
		basicstyle=\ttfamily\scriptsize,
		numbers=left,
		numberstyle=\tiny\color{gray},
		backgroundcolor=\color{codeBackground},
		commentstyle=\color{comentarios},
		keywordstyle=\color{identificador},
		stringstyle=\color{cadena},
		emph={pinMode,digitalWrite,tone,delay,HIGH,LOW},
		emphstyle={\color{blue}\bfseries},
		frame=single,
		frameround=tttt,
		rulecolor=\color{arduinoBlue},
		xleftmargin=5mm,
		xrightmargin=2mm,
		breaklines=false,
		columns=fullflexible,
		keepspaces=true,
		basewidth=0.5em 
]
void toneLed(int ledState, int frequency, 
		int duration) {
	digitalWrite(ledMusic, ledState);   
	tone(tonePin, frequency, duration); 
	delay(duration);                    
	digitalWrite(ledMusic, LOW);        
}
	\end{lstlisting}
	\caption{Funcion de Control LED-Tono}
	\label{fig:toneled-function}
\end{figure}

\subsection{Implementacion del Tema de Tetris}
\begin{figure}[H]
	\centering
	\begin{lstlisting}[
		language=C++,
		basicstyle=\ttfamily\scriptsize,
		numbers=left,
		numberstyle=\tiny\color{gray},
		backgroundcolor=\color{codeBackground},
		commentstyle=\color{comentarios},
		keywordstyle=\color{identificador},
		stringstyle=\color{cadena},
		emph={pinMode,digitalWrite,tone,delay,HIGH,LOW},
		emphstyle={\color{blue}\bfseries},
		frame=single,
		frameround=tttt,
		rulecolor=\color{arduinoBlue},
		xleftmargin=5mm,
		xrightmargin=2mm,
		breaklines=false,
		columns=fullflexible,
		keepspaces=true,
		basewidth=0.5em 
]
void tetrisTheme() {
	toneLed(HIGH, 659, 250);  
	toneLed(HIGH, 494, 125);  
	toneLed(HIGH, 523, 125);  
	toneLed(HIGH, 587, 250);  
	toneLed(HIGH, 523, 125);  
	toneLed(HIGH, 494, 125);  
	toneLed(HIGH, 440, 250);  
	toneLed(HIGH, 440, 125);  
	toneLed(HIGH, 523, 125);  
	toneLed(HIGH, 659, 250);  
	toneLed(HIGH, 587, 125);  
	toneLed(HIGH, 523, 125);  
	toneLed(HIGH, 494, 375);  
	toneLed(HIGH, 523, 125);  
	toneLed(HIGH, 587, 250);  
	toneLed(HIGH, 659, 250);  
	toneLed(HIGH, 523, 250);  
	toneLed(HIGH, 440, 250);  
	toneLed(HIGH, 440, 500);  
	
	delay(250);  
	
	toneLed(HIGH, 494, 250);  
	toneLed(HIGH, 587, 250);  
	toneLed(HIGH, 659, 250);  
	toneLed(HIGH, 698, 250);  
	toneLed(HIGH, 659, 125);  
	toneLed(HIGH, 587, 125);  
	toneLed(HIGH, 523, 375);  
	toneLed(HIGH, 523, 125);  
	toneLed(HIGH, 587, 250);  
	toneLed(HIGH, 659, 250);  
	toneLed(HIGH, 523, 250);  
	toneLed(HIGH, 494, 250);  
	toneLed(HIGH, 440, 500);  
	
	delay(500);  
}
	\end{lstlisting}
	\caption{Implementacion del Tema de Tetris}
	\label{fig:tetris-theme}
\end{figure}

\subsection{Tabla de Frecuencias}
\begin{table}*[H]
	\centering
	\begin{tabular}{|l|c|l|}
		\hline
		\textbf{Nota} & \textbf{Frecuencia (Hz)} & \textbf{Pin de Control} \\
		\hline
		Do (C4) & 262 & 13 \\
		Re (D4) & 294 & 12 \\
		Mi (E4) & 330 & 11 \\
		Fa (F4) & 349 & 10 \\
		Sol (G4) & 392 & 9 \\
		La (A4) & 440 & 8 \\
		Si (B4) & 494 & 7 \\
		Do (C5) & 523 & 6 \\
		Tetris Theme & Variable & 5 \\
		\hline
	\end{tabular}
	\caption{Asignacion de Notas y Pines de Control}
	\label{tab:frequency-pins}
\end{table}

\subsection{Control de Tiempo y Sincronizacion}
\begin{figure}[H]
	\centering
	\begin{lstlisting}[
		language=C++,
		basicstyle=\ttfamily\scriptsize,
		numbers=left,
		numberstyle=\tiny\color{gray},
		backgroundcolor=\color{codeBackground},
		commentstyle=\color{comentarios},
		keywordstyle=\color{identificador},
		stringstyle=\color{cadena},
		emph={delay,tone},
		emphstyle={\color{blue}\bfseries},
		frame=single,
		frameround=tttt,
		rulecolor=\color{arduinoBlue},
		xleftmargin=5mm,
		xrightmargin=2mm,
		breaklines=false,
		columns=fullflexible,
		keepspaces=true,
		basewidth=0.5em 
]
delay(100);    
delay(250);    
delay(500);    

tone(tonePin, frequency, duration);  
	\end{lstlisting}
	\caption{Mecanismos de Control Temporal}
	\label{fig:timing-control}
\end{figure}

\subsubsection{Estructura del Tema Musical}
\begin{figure}[H]
	\centering
	\begin{lstlisting}[
		language=C++,
		basicstyle=\ttfamily\scriptsize,
		numbers=left,
		numberstyle=\tiny\color{gray},
		backgroundcolor=\color{codeBackground},
		commentstyle=\color{comentarios},
		keywordstyle=\color{identificador},
		stringstyle=\color{cadena},
		emph={toneLed,delay,HIGH},
		emphstyle={\color{blue}\bfseries},
		frame=single,
		frameround=tttt,
		rulecolor=\color{arduinoBlue},
		xleftmargin=5mm,
		xrightmargin=2mm,
		breaklines=false,
		columns=fullflexible,
		keepspaces=true,
		basewidth=0.5em 
]
toneLed(HIGH, 659, 250);  
toneLed(HIGH, 494, 125);  
toneLed(HIGH, 523, 125);  

toneLed(HIGH, 587, 250);  
toneLed(HIGH, 523, 125);  
toneLed(HIGH, 494, 125);  

toneLed(HIGH, 494, 250);  
toneLed(HIGH, 587, 250);  
toneLed(HIGH, 659, 250);  
	\end{lstlisting}
	\caption{Estructura Musical del Tema de Tetris}
	\label{fig:music-structure}
\end{figure}

\subsection{Diagrama de Conexiones}
\begin{figure}[H]
	\centering
	\begin{lstlisting}[
		language=C++,
		basicstyle=\ttfamily\scriptsize,
		numbers=left,
		numberstyle=\tiny\color{gray},
		backgroundcolor=\color{codeBackground},
		frame=single,
		frameround=tttt,
		rulecolor=\color{arduinoBlue},
		xleftmargin=5mm,
		xrightmargin=2mm,
		breaklines=false,
		columns=fullflexible,
		keepspaces=true,
		basewidth=0.5em 
]
Arduino    |    Componente
-----------------------
Pin 4      ->   Buzzer (+)
GND        ->   Buzzer (-)
Pin 3      ->   LED Musical (+)
GND        ->   LED Musical (-)
Pin 13     ->   Boton Do (C4)
Pin 12     ->   Boton Re (D4)
Pin 11     ->   Boton Mi (E4)
Pin 10     ->   Boton Fa (F4)
Pin 9      ->   Boton Sol (G4)
Pin 8      ->   Boton La (A4)
Pin 7      ->   Boton Si (B4)
Pin 6      ->   Boton Do (C5)
Pin 5      ->   Boton Tetris
	\end{lstlisting}
	\caption{Diagrama de Conexiones del Sistema}
	\label{fig:connection-diagram}
\end{figure}

\section{Detalles de Implementación}

\subsection{Estructura de Datos}
La implementación utiliza estructuras de datos simples y eficientes para el manejo de las notas musicales. El siguiente código muestra la organización básica:

\begin{figure}[H]
\centering
\begin{lstlisting}[
	language=C++,
	basicstyle=\ttfamily\scriptsize,
	numbers=left,
	numberstyle=\tiny\color{gray},
	backgroundcolor=\color{codeBackground},
	frame=single,
	frameround=tttt,
	rulecolor=\color{arduinoBlue},
	xleftmargin=5mm,
	xrightmargin=2mm,
	breaklines=false,
	columns=fullflexible,
	keepspaces=true,
	basewidth=0.5em 
]
const int NOTAS_BASICAS = 8;
const int FRECUENCIAS[NOTAS_BASICAS] = {262, 294, 330, 349, 392, 440, 494, 523};
const int PINES_BOTONES[NOTAS_BASICAS] = {13, 12, 11, 10, 9, 8, 7, 6};
\end{lstlisting}
\caption{Definición de Arrays para Notas y Pines}
\label{fig:arrays-definition}
\end{figure}

\subsection{Control de Tiempo Avanzado}
El sistema implementa un control de tiempo preciso para la reproducción musical. A continuación se muestra la implementación detallada:

\begin{figure}[H]
\centering
\begin{lstlisting}[
	language=C++,
	basicstyle=\ttfamily\scriptsize,
	numbers=left,
	numberstyle=\tiny\color{gray},
	backgroundcolor=\color{codeBackground},
	frame=single,
	frameround=tttt,
	rulecolor=\color{arduinoBlue},
	xleftmargin=5mm,
	xrightmargin=2mm
]
unsigned long previousMillis = 0;
const long interval = 100;

void timedTone(int frequency, int duration) {
unsigned long currentMillis = millis();
if (currentMillis - previousMillis >= interval) {
	previousMillis = currentMillis;
	tone(tonePin, frequency, duration);
}
}
\end{lstlisting}
\caption{Implementación de Control de Tiempo}
\label{fig:time-control}
\end{figure}

\subsection{Sistema de Efectos LED}
El sistema de iluminación LED se ha implementado con diferentes patrones para indicar el estado del piano:

\begin{figure}[H]
\centering
\begin{lstlisting}[
	language=C++,
	basicstyle=\ttfamily\scriptsize,
	numbers=left,
	numberstyle=\tiny\color{gray},
	backgroundcolor=\color{codeBackground},
	frame=single,
	frameround=tttt,
	rulecolor=\color{arduinoBlue},
	xleftmargin=5mm,
	xrightmargin=2mm
]
void ledPattern(int pattern) {
	switch(pattern) {
		case 0:  
		digitalWrite(ledMusic, HIGH);
		delay(50);
		digitalWrite(ledMusic, LOW);
		break;
		case 1:  
		for(int i = 0; i < 3; i++) {
			digitalWrite(ledMusic, HIGH);
			delay(100);
			digitalWrite(ledMusic, LOW);
			delay(100);
		}
		break;
	}
}
\end{lstlisting}
\caption{Sistema de Patrones LED}
\label{fig:led-patterns}
\end{figure}

\subsection{Optimización de Memoria}
La siguiente implementación muestra cómo se ha optimizado el uso de memoria:

\begin{figure}[H]
\centering
\begin{lstlisting}[
	language=C++,
	basicstyle=\ttfamily\scriptsize,
	numbers=left,
	numberstyle=\tiny\color{gray},
	backgroundcolor=\color{codeBackground},
	frame=single,
	frameround=tttt,
	rulecolor=\color{arduinoBlue},
	xleftmargin=5mm,
	xrightmargin=2mm
]
struct NotaMusical {
	uint16_t frecuencia;
	uint8_t duracion;
};

const PROGMEM NotaMusical tetrisNotas[] = {
	{659, 250}, {494, 125}, {523, 125},
	{587, 250}, {523, 125}, {494, 125},
	{440, 250}, {440, 125}, {523, 125}
};
\end{lstlisting}
\caption{Estructuras de Datos Optimizadas}
\label{fig:optimized-structures}
\end{figure}

\section{Extensiones del Sistema}

\subsection{Modo de Práctica}
Se implementó un modo de práctica con el siguiente código:

\begin{figure}[H]
\centering
\begin{lstlisting}[
	language=C++,
	basicstyle=\ttfamily\scriptsize,
	numbers=left,
	numberstyle=\tiny\color{gray},
	backgroundcolor=\color{codeBackground},
	frame=single,
	frameround=tttt,
	rulecolor=\color{arduinoBlue},
	xleftmargin=5mm,
	xrightmargin=2mm
]
void modoPractica() {
static uint8_t notaActual = 0;
static unsigned long tiempoInicio = 0;

if (millis() - tiempoInicio > 2000) {
	tiempoInicio = millis();
	toneLed(HIGH, FRECUENCIAS[notaActual], 500);
	notaActual = (notaActual + 1) % NOTAS_BASICAS;
}
}
\end{lstlisting}
\caption{Implementación del Modo Práctica}
\label{fig:practice-mode}
\end{figure}

\subsection{Sistema de Detección de Errores}
Se implementó un sistema básico de detección de errores:

\begin{figure}[H]
\centering
\begin{lstlisting}[
	language=C++,
	basicstyle=\ttfamily\scriptsize,
	numbers=left,
	numberstyle=\tiny\color{gray},
	backgroundcolor=\color{codeBackground},
	frame=single,
	frameround=tttt,
	rulecolor=\color{arduinoBlue},
	xleftmargin=5mm,
	xrightmargin=2mm
]
bool verificarSistema() {
	bool sistemaCorrecto = true;
	
	for (int i = 0; i < NOTAS_BASICAS; i++) {
		pinMode(PINES_BOTONES[i], INPUT);
		if (digitalRead(PINES_BOTONES[i]) == HIGH) {
			sistemaCorrecto = false;
			ledPattern(1);
		}
	}
	
	return sistemaCorrecto;
}
\end{lstlisting}
\caption{Sistema de Verificación}
\label{fig:error-detection}
\end{figure}

\section{Conclusiones y Recomendaciones}

\subsection{Resumen de Características}
El piano digital implementado ofrece las siguientes características principales:
\begin{itemize}
\item 8 notas musicales básicas
\item Tema musical de Tetris pregrabado
\item Retroalimentación visual mediante LED
\item Sistema de control de tiempo preciso
\item Modo de práctica
\item Sistema de detección de errores
\end{itemize}

\subsection{Recomendaciones de Uso}
Para un funcionamiento óptimo del sistema, se recomienda:
\begin{itemize}
\item Verificar las conexiones antes de cada uso
\item Mantener un tiempo mínimo entre pulsaciones de botones
\item Utilizar una fuente de alimentación estable
\item Realizar pruebas periódicas del sistema de verificación
\end{itemize}