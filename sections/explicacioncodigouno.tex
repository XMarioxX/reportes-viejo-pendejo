% explicaicon_codigo.tex

% Configuración preliminar para ajustar el espacio
\setlength{\parindent}{0pt}
\setlength{\parskip}{6pt}



\begin{figure}[H]
	\centering
	\begin{lstlisting}[
		language=C++,
		basicstyle=\ttfamily\scriptsize,
		numbers=left,
		numberstyle=\tiny\color{gray},
		backgroundcolor=\color{codeBackground},
		commentstyle=\color{comentarios},
		keywordstyle=\color{identificador},
		stringstyle=\color{cadena},
		emph={pinMode,digitalWrite,HIGH,LOW},
		emphstyle={\color{blue}\bfseries},
		frame=single,
		frameround=tttt,
		rulecolor=\color{arduinoBlue},
		breaklines=true,   % Permite la ruptura de líneas largas
		columns=fullflexible,
		keepspaces=true,
		basewidth=0.5em,
		xleftmargin=5mm,   % Márgenes laterales ajustados
		xrightmargin=0mm
		]
void loop() {
	bool anyKeyPressed = false;
	
	// Piano keys functionality
	if (digitalRead(13) == HIGH) {
		stopCurrentMelody();
		tone(tonePin, 262, 100);  // C4 (Do)
		anyKeyPressed = true;
		lastToneTime = millis();
	}
	if (digitalRead(12) == HIGH) {
		stopCurrentMelody();
		tone(tonePin, 294, 100);  //D4 (Re)
		anyKeyPressed = true;
		lastToneTime = millis();
	}
	\end{lstlisting}
	\caption{Función loop y condicionales}
	\label{fig:codigo-led}
\end{figure}
La figura 3  implementa la funcionalidad de un teclado de piano básico en el loop(). Define una variable anyKeyPressed para indicar si alguna tecla ha sido presionada.

Si el pin 13 detecta una señal alta (HIGH), el programa detiene la melodía actual (con stopCurrentMelody()), reproduce la nota C4 (Do) en el pin de tono (tonePin) a 262 Hz durante 100 ms, marca anyKeyPressed como verdadero, y actualiza el tiempo de la última nota tocada (lastToneTime). Del mismo modo, si el pin 12 está en HIGH, detiene la melodía, reproduce la nota D4 (Re) a 294 Hz durante 100 ms, y actualiza las mismas variables.
