% desarrollo_practica.tex

\section{Desarrollo de la Practica}

\subsection{Material Utilizado}
\begin{componentBox}{Lista de Componentes}
	\begin{itemize}[leftmargin=*,itemsep=1pt,parsep=1pt]
		\item 1 Placa Arduino UNO
		\item 9 Botones pulsadores
		\item 9 LEDs
		\item 9 Resistencias de \SI{1}{\kilo\ohm} para LEDs
		\item 9 Resistencias de \SI{10}{\kilo\ohm} para pull-up
		\item 1 Buzzer pasivo
		\item Cables jumper
		\item 1 Protoboard
	\end{itemize}
\end{componentBox}

\subsection{Esquematico del Circuito}
\begin{componentBox}{Conexiones Principales}
	[Incluir imagen del esquematico]
	\begin{itemize}[leftmargin=*,itemsep=1pt,parsep=1pt]
		\item \textbf{LEDs}: Conectados a pines 11-19
		\item \textbf{Botones}: Conectados a pines 2-10
		\item \textbf{Buzzer}: Conectado al pin PWM 20
	\end{itemize}
\end{componentBox}

\subsection{Explicacion de las Conexiones}
\begin{componentBox}{Detalles de Conexion}
	\begin{itemize}[leftmargin=*,itemsep=1pt,parsep=1pt]
		\item \textbf{Conexion de LEDs}:
		\begin{itemize}[itemsep=0pt,parsep=0pt]
			\item Anodo -> Resistencia \SI{1}{\kilo\ohm} -> Pin Arduino
			\item Catodo -> GND
			\item Resistencia limita corriente a \SI{20}{\milli\ampere}
		\end{itemize}
		
		\item \textbf{Conexion de Botones}:
		\begin{itemize}[itemsep=0pt,parsep=0pt]
			\item Terminal 1 -> Pin Arduino
			\item Terminal 2 -> GND
			\item Resistencia pull-up interna habilitada
		\end{itemize}
		
		\item \textbf{Conexion del Buzzer}:
		\begin{itemize}[itemsep=0pt,parsep=0pt]
			\item Terminal positivo -> Pin PWM Arduino
			\item Terminal negativo -> GND
			\item No requiere resistencia limitadora
		\end{itemize}
	\end{itemize}
\end{componentBox}

\subsection{Descripcion del Codigo}
\begin{componentBox}{Estructura del Programa}
	\begin{itemize}[leftmargin=*,itemsep=1pt,parsep=1pt]
		\item \textbf{Variables Globales}:
		\begin{itemize}[itemsep=0pt,parsep=0pt]
			\item Arrays para pines de botones y LEDs
			\item Constantes para frecuencias de notas
			\item Pin designado para el buzzer
		\end{itemize}
		
		\item \textbf{Funcion setup()}:
		\begin{itemize}[itemsep=0pt,parsep=0pt]
			\item Configura pines de botones como INPUT\_PULLUP
			\item Configura pines de LEDs como OUTPUT
			\item Inicializa pin del buzzer
		\end{itemize}
		
		\item \textbf{Funcion loop()}:
		\begin{itemize}[itemsep=0pt,parsep=0pt]
			\item Lee estado de botones
			\item Genera tonos correspondientes
			\item Controla LEDs asociados
		\end{itemize}
		
		\item \textbf{Funcion playMelody()}:
		\begin{itemize}[itemsep=0pt,parsep=0pt]
			\item Implementa melodia especial
			\item Control de tiempos y secuencias
		\end{itemize}
	\end{itemize}
\end{componentBox}

\subsection{Observaciones y Comentarios}
\begin{componentBox}{Consideraciones Importantes}
	\begin{itemize}[leftmargin=*,itemsep=1pt,parsep=1pt]
		\item El uso de resistencias pull-up internas simplifica el circuito
		\item La funcion tone() bloquea algunas interrupciones
		\item Se recomienda implementar debounce en los botones
		\item La melodia puede personalizarse segun necesidades
		\item Los LEDs proporcionan retroalimentacion visual util
		\item El codigo es escalable para mas notas/botones
	\end{itemize}
	
	\begin{itemize}[leftmargin=*,itemsep=1pt,parsep=1pt]
		\item \textbf{Mejoras Posibles}:
		\begin{itemize}[itemsep=0pt,parsep=0pt]
			\item Implementar control de volumen
			\item Agregar mas melodias predefinidas
			\item Mejorar el manejo de multiples botones
			\item Anadir efectos de sonido adicionales
		\end{itemize}
	\end{itemize}
\end{componentBox}