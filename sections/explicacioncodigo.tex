% explicaicon_codigo.tex

% Configuración preliminar para ajustar el espacio
\setlength{\parindent}{0pt}
\setlength{\parskip}{6pt}
\section{Explicación del Código}

\begin{figure}[H]
	\centering
	\begin{lstlisting}[
		language=C++,
		basicstyle=\ttfamily\scriptsize,
		numbers=left,
		numberstyle=\tiny\color{gray},
		backgroundcolor=\color{codeBackground},
		commentstyle=\color{comentarios},
		keywordstyle=\color{identificador},
		stringstyle=\color{cadena},
		emph={pinMode,digitalWrite,HIGH,LOW},
		emphstyle={\color{blue}\bfseries},
		frame=single,
		frameround=tttt,
		rulecolor=\color{arduinoBlue},
		breaklines=true,   % Permite la ruptura de líneas largas
		columns=fullflexible,
		keepspaces=true,
		basewidth=0.5em,
		xleftmargin=5mm,   % Márgenes laterales ajustados
		xrightmargin=0mm
		]
int tonePin = 4;
int ledMusic = 3;
unsigned long lastToneTime = 0;
bool isPlaying = false;
volatile bool shouldStopMelody = false;

void setup() {
	for (int pin = 5; pin <= 13; pin++) {
		pinMode(pin, INPUT);
	}
	pinMode(tonePin, OUTPUT);
	pinMode(ledMusic, OUTPUT);
}
	\end{lstlisting}
	\caption{Variables globales y función setup}
	\label{fig:codigo-led}
\end{figure}
La figura 2 muestra la configuración para el pin 4 para emitir sonidos y el pin 3 para controlar un LED indicador de música. En la función setup(), establece los pines 5 a 13 como entradas y define los pines tonePin y ledMusic como salidas para controlar el sonido y el LED. Además, usa variables para llevar el control del tiempo de reproducción y permite detener la melodía si es necesario.
