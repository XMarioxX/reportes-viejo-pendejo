% introduccion_componentes.tex

% Configuración preliminar para ajustar el espacio
\setlength{\parindent}{0pt}
\setlength{\parskip}{6pt}

\section{Introducción}
Esta práctica se centra en la creación de un piano básico utilizando Arduino, donde aprenderemos a integrar múltiples componentes electrónicos para crear un sistema interactivo musical \citep{monk2017programming}. Los pianos electrónicos modernos utilizan circuitos y componentes para generar distintas notas musicales, y en esta práctica simularemos este funcionamiento de manera simplificada \citep{evans2011arduino}.


\subsection{LEDs (Diodos Emisores de Luz)}
\begin{componentBox}{Características y Conexión \citep{platt2014encyclopedia}}
	\begin{itemize}[leftmargin=*,itemsep=1pt,parsep=1pt]
		\item \textbf{Polaridad}:
		\begin{itemize}[itemsep=0pt,parsep=0pt]
			\item Ánodo (+): Pata más larga
			\item Cátodo (-): Pata más corta
		\end{itemize}
		\item \textbf{Especificaciones}:
		\begin{itemize}[itemsep=0pt,parsep=0pt]
			\item Voltaje típico: \SI{2.0}{\volt} - \SI{3.3}{\volt}
			\item Corriente: \SI{20}{\milli\ampere}
		\end{itemize}
		\item \textbf{Conexión}:
		\begin{itemize}[itemsep=0pt,parsep=0pt]
			\item Ánodo → Resistencia → Pin
			\item Cátodo → GND
		\end{itemize}
	\end{itemize}
\end{componentBox}

\begin{figure}[H]
	\centering
	\begin{lstlisting}[
		language=C++,
		basicstyle=\ttfamily\scriptsize,
		numbers=left,
		numberstyle=\tiny\color{gray},
		backgroundcolor=\color{codeBackground},
		commentstyle=\color{comentarios},
		keywordstyle=\color{identificador},
		stringstyle=\color{cadena},
		emph={pinMode,digitalWrite,HIGH,LOW},
		emphstyle={\color{blue}\bfseries},
		frame=single,
		frameround=tttt,
		rulecolor=\color{arduinoBlue},
		xleftmargin=5mm,
		xrightmargin=2mm,
		breaklines=false,
		columns=fullflexible,
		keepspaces=true
		basewidth=0.5em 
]
const int ledPin = 13;

void setup() {
	pinMode(ledPin, OUTPUT);
}

void loop() {
	digitalWrite(ledPin, HIGH);  // Encender
	delay(1000);
	digitalWrite(ledPin, LOW);   // Apagar
	delay(1000);
}
	\end{lstlisting}
	\caption{Código de control de LED en Arduino}
	\label{fig:codigo-led}
\end{figure}

\subsection{Botones Pulsadores}
\begin{componentBox}{Características y Conexión \citep{scherz2016practical}}
	\begin{itemize}[leftmargin=*,itemsep=1pt,parsep=1pt]
		\item \textbf{Tipo}: Interruptores (NO)
		\item \textbf{Características}:
		\begin{itemize}[itemsep=0pt,parsep=0pt]
			\item Sin polaridad específica
			\item Requieren resistencia pull-up
			\item Estado normal: Abierto
		\end{itemize}
		\item \textbf{Conexión}:
		\begin{itemize}[itemsep=0pt,parsep=0pt]
			\item Terminal 1 → Pin Arduino
			\item Terminal 2 → GND
			\item R pull-up \SI{10}{\kilo\ohm} → \SI{5}{\volt}
		\end{itemize}
	\end{itemize}
\end{componentBox}

\subsection{Buzzer Pasivo}
\begin{componentBox}{Características y Operación \citep{arduino_tone}}
	\begin{itemize}[leftmargin=*,itemsep=1pt,parsep=1pt]
		\item \textbf{Características}:
		\begin{itemize}[itemsep=0pt,parsep=0pt]
			\item Sin oscilador interno
			\item Requiere señal PWM
			\item Voltaje: \SI{3}{\volt}-\SI{12}{\volt}
		\end{itemize}
		\item \textbf{Conexión}:
		\begin{itemize}[itemsep=0pt,parsep=0pt]
			\item (+) → Pin PWM Arduino
			\item (-) → GND
		\end{itemize}
		\item \textbf{Notas musicales}:
		\begin{multicols}{2}
			\begin{itemize}[itemsep=0pt,parsep=0pt]
				\item DO (C4): \SI{261.63}{\hertz}
				\item RE (D4): \SI{293.66}{\hertz}
				\item MI (E4): \SI{329.63}{\hertz}
				\item FA (F4): \SI{349.23}{\hertz}
				\item SOL (G4): \SI{392.00}{\hertz}
				\item LA (A4): \SI{440.00}{\hertz}
				\item SI (B4): \SI{493.88}{\hertz}
				\item DO (C5): \SI{523.25}{\hertz}
			\end{itemize}
		\end{multicols}
	\end{itemize}
\end{componentBox}

\subsection{Resistencias}
\begin{componentBox}{Tipos y Usos \citep{platt2014encyclopedia}}
	\begin{itemize}[leftmargin=*,itemsep=1pt,parsep=1pt]
		\item \textbf{Para LEDs}:
		\begin{itemize}[itemsep=0pt,parsep=0pt]
			\item Rango: \SI{220}{\ohm} - \SI{1}{\kilo\ohm}
			\item Limitan corriente
			\item R = $\frac{V_{\text{fuente}} - V_{\text{led}}}{I_{\text{led}}}$
		\end{itemize}
		\item \textbf{Pull-up}:
		\begin{itemize}[itemsep=0pt,parsep=0pt]
			\item Valor: \SI{10}{\kilo\ohm}
			\item Mantiene estado lógico
		\end{itemize}
		\item \textbf{Código de colores}:
		\begin{itemize}[itemsep=0pt,parsep=0pt]
			\item \SI{220}{\ohm}: Rojo-Rojo-Marrón
			\item \SI{1}{\kilo\ohm}: Marrón-Negro-Rojo
			\item \SI{10}{\kilo\ohm}: Marrón-Negro-Naranja
		\end{itemize}
	\end{itemize}
\end{componentBox}

\section{Consideraciones de Diseño}
\begin{componentBox}{Recomendaciones Importantes \citep{banzi2009getting}}
	\begin{itemize}[leftmargin=*,itemsep=1pt,parsep=1pt]
		\item \textbf{Pines Arduino}:
		\begin{itemize}[itemsep=0pt,parsep=0pt]
			\item PWM (3,5,6,9,10,11) → buzzer
			\item LEDs en pines consecutivos
			\item Botones con pull-up interno
		\end{itemize}
		\item \textbf{Seguridad}:
		\begin{itemize}[itemsep=0pt,parsep=0pt]
			\item Verificar polaridad
			\item Máx. \SI{40}{\milli\ampere}/pin
			\item Resistencias adecuadas
			\item Desconectar al modificar
		\end{itemize}
		\item \textbf{Código}:
		\begin{itemize}[itemsep=0pt,parsep=0pt]
			\item Constantes para pines/notas
			\item Debounce en botones
			\item Funciones estructuradas
		\end{itemize}
	\end{itemize}
\end{componentBox}

\section{Objetivo del Proyecto}
\begin{componentBox}{Objetivos}
	El objetivo es crear un piano digital funcional que integre múltiples componentes electrónicos \citep{evans2011arduino}. Se desarrollarán habilidades en:
	\begin{itemize}[leftmargin=*,itemsep=1pt]
		\item Manejo de entradas/salidas digitales
		\item Generación de tonos con PWM
		\item Interacciones usuario-dispositivo
		\item Sistemas multicomponente
	\end{itemize}
\end{componentBox}