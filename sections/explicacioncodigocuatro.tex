% explicaicon_codigo.tex

% Configuración preliminar para ajustar el espacio
\setlength{\parindent}{0pt}
\setlength{\parskip}{6pt}
\begin{figure}[H]
	\centering
	\begin{lstlisting}[
		language=C++,
		basicstyle=\ttfamily\scriptsize,
		numbers=left,
		numberstyle=\tiny\color{gray},
		backgroundcolor=\color{codeBackground},
		commentstyle=\color{comentarios},
		keywordstyle=\color{identificador},
		stringstyle=\color{cadena},
		emph={pinMode,digitalWrite,HIGH,LOW},
		emphstyle={\color{blue}\bfseries},
		frame=single,
		frameround=tttt,
		rulecolor=\color{arduinoBlue},
		breaklines=true,   % Permite la ruptura de líneas largas
		columns=fullflexible,
		keepspaces=true,
		basewidth=0.5em,
		xleftmargin=5mm,   % Márgenes laterales ajustados
		xrightmargin=0mm
		]

	if (!anyKeyPressed && (millis() - lastToneTime > 100)) {
		digitalWrite(ledMusic, LOW);
		isPlaying = false;
	}
	
	delay(10);  
	}
	
		
	\end{lstlisting}
	\caption{Condicionales}
	\label{fig:codigo-led}
\end{figure}
La figura 6 muestra el apagado del LED indicador de música si no se está presionando ninguna tecla y han pasado al menos 100 ms desde la última nota tocada.

Primero, verifica si no hay teclas presionadas (!anyKeyPressed) y si el tiempo transcurrido desde la última nota (millis() - lastToneTime) supera los 100 ms. Si ambas condiciones se cumplen, apaga el LED (digitalWrite(ledMusic, LOW)) y marca que la música no está sonando (isPlaying = false). Finalmente, incluye un pequeño retraso de 10 ms para estabilizar el ciclo del loop().