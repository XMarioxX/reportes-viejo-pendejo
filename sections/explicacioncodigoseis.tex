% explicaicon_codigo.tex

% Configuración preliminar para ajustar el espacio
\setlength{\parindent}{0pt}
\setlength{\parskip}{6pt}
\begin{figure}[H]
	\centering
	\begin{lstlisting}[
		language=C++,
		basicstyle=\ttfamily\scriptsize,
		numbers=left,
		numberstyle=\tiny\color{gray},
		backgroundcolor=\color{codeBackground},
		commentstyle=\color{comentarios},
		keywordstyle=\color{identificador},
		stringstyle=\color{cadena},
		emph={pinMode,digitalWrite,HIGH,LOW},
		emphstyle={\color{blue}\bfseries},
		frame=single,
		frameround=tttt,
		rulecolor=\color{arduinoBlue},
		breaklines=true,   % Permite la ruptura de líneas largas
		columns=fullflexible,
		keepspaces=true,
		basewidth=0.5em,
		xleftmargin=5mm,   % Márgenes laterales ajustados
		xrightmargin=0mm
		]
		
void tetrisTheme() {
	// Theme A
	toneLed(659, 250);  // E5
	if (shouldStopMelody) return;
	toneLed(494, 125);  // B4
	if (shouldStopMelody) return;
	toneLed(523, 125);  // C5
	if (shouldStopMelody) return;
	toneLed(587, 250);  // D5
	if (shouldStopMelody) return;
	toneLed(523, 125);  // C5
	if (shouldStopMelody) return;
	toneLed(494, 125);  // B4
	if (shouldStopMelody) return;
	toneLed(440, 250);  // A4
	if (shouldStopMelody) return;
	toneLed(440, 125);  // A4
	if (shouldStopMelody) return;
	toneLed(523, 125);  // C5
	if (shouldStopMelody) return;
	toneLed(659, 250);  // E5
	if (shouldStopMelody) return;
	toneLed(587, 125);  // D5
	if (shouldStopMelody) return;
	toneLed(523, 125);  // C5
	if (shouldStopMelody) return;
	toneLed(494, 375);  // B4
	if (shouldStopMelody) return;
	toneLed(523, 125);  // C5
	if (shouldStopMelody) return;
	toneLed(587, 250);  // D5
	if (shouldStopMelody) return;
	toneLed(659, 250);  // E5
	if (shouldStopMelody) return;
	toneLed(523, 250);  // C5
	if (shouldStopMelody) return;
	toneLed(440, 250);  // A4
	if (shouldStopMelody) return;
	toneLed(440, 500);  // A4
	if (shouldStopMelody) return;
	
	delay(250);  
	if (shouldStopMelody) return;
		
		
		
	\end{lstlisting}
	\caption{Función tetrisTheme() y condicionales}
	\label{fig:codigo-led}
\end{figure}
El código de la figura 8 define la función tetrisTheme() que reproduce el tema de Tetris (Theme A) utilizando tonos y frecuencias específicos. Cada llamada a toneLed() reproduce una nota musical en el pin designado (tonePin) con una duración específica, representando una parte de la melodía.

La función empieza reproduciendo una secuencia de notas con sus respectivas frecuencias y duraciones, como E5, B4, C5, y así sucesivamente. Después de cada nota, verifica si se ha activado shouldStopMelody, lo cual indica si la melodía debe detenerse y, de ser así, interrumpe la función.

La melodía incluye una breve pausa de 250 ms al final para permitir un intervalo entre repeticiones, haciendo que el tema se sienta más fluido y dándole tiempo al usuario antes de la próxima iteración.