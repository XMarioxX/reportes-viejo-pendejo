% explicaicon_codigo.tex

% Configuración preliminar para ajustar el espacio
\setlength{\parindent}{0pt}
\setlength{\parskip}{6pt}
\begin{figure}[H]
	\centering
	\begin{lstlisting}[
		language=C++,
		basicstyle=\ttfamily\scriptsize,
		numbers=left,
		numberstyle=\tiny\color{gray},
		backgroundcolor=\color{codeBackground},
		commentstyle=\color{comentarios},
		keywordstyle=\color{identificador},
		stringstyle=\color{cadena},
		emph={pinMode,digitalWrite,HIGH,LOW},
		emphstyle={\color{blue}\bfseries},
		frame=single,
		frameround=tttt,
		rulecolor=\color{arduinoBlue},
		breaklines=true,   % Permite la ruptura de líneas largas
		columns=fullflexible,
		keepspaces=true,
		basewidth=0.5em,
		xleftmargin=5mm,   % Márgenes laterales ajustados
		xrightmargin=0mm
		]
		
	// Theme B
	toneLed(494, 250);  // B4
	if (shouldStopMelody) return;
	toneLed(587, 250);  // D5
	if (shouldStopMelody) return;
	toneLed(659, 250);  // E5
	if (shouldStopMelody) return;
	toneLed(698, 250);  // F5
	if (shouldStopMelody) return;
	toneLed(659, 125);  // E5
	if (shouldStopMelody) return;
	toneLed(587, 125);  // D5
	if (shouldStopMelody) return;
	toneLed(523, 375);  // C5
	if (shouldStopMelody) return;
	toneLed(523, 125);  // C5
	if (shouldStopMelody) return;
	toneLed(587, 250);  // D5
	if (shouldStopMelody) return;
	toneLed(659, 250);  // E5
	if (shouldStopMelody) return;
	toneLed(523, 250);  // C5
	if (shouldStopMelody) return;
	toneLed(494, 250);  // B4
	if (shouldStopMelody) return;
	toneLed(440, 500);  // A4
	
	delay(500);  // Longer pause at the end
	}
\end{lstlisting}
\caption{Función tetrisTheme() y condicionales}
\label{fig:codigo-led}
\end{figure}
La figura 9 amplía la función tetrisTheme() en Arduino para incluir el "Theme B" de la melodía de Tetris. La función utiliza llamadas consecutivas a toneLed() para reproducir notas como B4, D5, E5, F5 y otras, cada una con una duración determinada. Después de cada nota, verifica la variable shouldStopMelody para ver si la reproducción debe detenerse; si es así, la función finaliza inmediatamente.

Al final de "Theme B", se agrega una pausa más larga de 500 ms para crear un descanso perceptible antes de cualquier repetición o cambio en la reproducción. Esto permite un flujo más natural entre los temas y evita que la melodía se sienta apresurada.