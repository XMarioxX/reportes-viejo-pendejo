% portada.tex
% Configuración del fondo solo para la portada
\backgroundsetup{
	scale=1,
	angle=0,
	opacity=0.2,
	contents={\includegraphics[width=\paperwidth,height=\paperheight]{images/fondo_portada.png}}
}
\BgThispage

\begin{center}
	\vspace*{2cm}
	
	\Large\textbf{Transistor BJT y MOSFET}
	
	\vspace{1.5cm}
	
	Autor(es):
	
	\vspace{0.5cm}
	
	Mario Eduardo Sánchez Mejía\\
	Fidel Alberto Zarco Áviles
	
	\vspace{0.3cm}
	
	\small{l21120721@morelia.tecnm.mx}\\
	\small{l20121258@morelia.tecnm.mx}
	
	\vspace{1.5cm}
	
	Asesor(@s):
	
	\vspace{0.5cm}
	
	Luis Ulises Chávez Campos
	
	\vspace{0.5cm}
	
	\begin{center}
		\begin{abstract}
			\noindent
			\justifying
			Se presenta el desarrollo e implementación de un sistema de control electrónico basado en transistores como interruptores, utilizando microcontroladores Arduino. La implementación incorpora componentes específicos como el transistor BD235 y el MOSFET IRLZ14, junto con potenciómetros, para gestionar cargas de alta potencia, particularmente bombillas de 12V. El desarrollo experimental facilita la comprensión práctica del comportamiento de transistores en aplicaciones de conmutación, además de demostrar la implementación efectiva de sistemas de control mediante señales tanto digitales como analógicas. La integración resultante evidencia la aplicación práctica de principios fundamentales en electrónica de potencia y programación de microcontroladores en sistemas embebidos.
			
			\vspace{0.5cm}
			\noindent
			\textbf{Palabras clave:} Arduino, Transistores, MOSFET, Sistemas de Control, Electrónica de Potencia
		\end{abstract}
	\end{center}
\end{center}

% Desactivar el fondo para las siguientes páginas
\clearpage
\backgroundsetup{contents={}}