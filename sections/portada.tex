% portada.tex
% Configuración del fondo solo para la portada
\backgroundsetup{
	scale=1,
	angle=0,
	opacity=0.2,
	contents={\includegraphics[width=\paperwidth,height=\paperheight]{images/fondo_portada.png}}
}
\BgThispage

\begin{center}
	\vspace*{2cm}
	
	\Large\textbf{Implementación del Buffer}
	
	\vspace{1.5cm}
	
	Autor(es):
	
	\vspace{0.5cm}
	
	Mario Eduardo Sánchez Mejía\\
	Fidel Alberto Zarco Áviles
	
	\vspace{0.3cm}
	
	\small{21120721@morelia.tecnm.mx}\\
	\small{l20121258@morelia.tecnm.mx}
	
	\vspace{1.5cm}
	
	Asesor(@s):
	
	\vspace{0.5cm}
	
	Luis Ulises Chávez Campos
	
	\vspace{0.5cm}
	
	\begin{abstract}
		\noindent
		\justifying
		Se desarrolla un sistema de piano digital mediante Arduino, integrando elementos electrónicos como buzzers, LEDs y botones pulsadores. El proyecto permite la comprensión de generación de tonos musicales a través de programación, empleando un buzzer pasivo y la función tone() de Arduino. El sistema implementa las ocho notas musicales básicas y una función para reproducir melodías MIDI convertidas. La práctica integra conceptos de electrónica digital, programación y principios musicales fundamentales.
		
		\vspace{0.5cm}
		\noindent
		\textbf{Palabras clave:} Arduino, Electrónica Digital, Programación, Piano Digital, Sistemas Programables
	\end{abstract}
\end{center}

% Desactivar el fondo para las siguientes páginas
\clearpage
\backgroundsetup{contents={}}