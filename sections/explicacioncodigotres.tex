% explicaicon_codigo.tex

% Configuración preliminar para ajustar el espacio
\setlength{\parindent}{0pt}
\setlength{\parskip}{6pt}
\begin{figure}[H]
	\centering
	\begin{lstlisting}[
		language=C++,
		basicstyle=\ttfamily\scriptsize,
		numbers=left,
		numberstyle=\tiny\color{gray},
		backgroundcolor=\color{codeBackground},
		commentstyle=\color{comentarios},
		keywordstyle=\color{identificador},
		stringstyle=\color{cadena},
		emph={pinMode,digitalWrite,HIGH,LOW},
		emphstyle={\color{blue}\bfseries},
		frame=single,
		frameround=tttt,
		rulecolor=\color{arduinoBlue},
		breaklines=true,   % Permite la ruptura de líneas largas
		columns=fullflexible,
		keepspaces=true,
		basewidth=0.5em,
		xleftmargin=5mm,   % Márgenes laterales ajustados
		xrightmargin=0mm
		]
	if (digitalRead(6) == HIGH) {
		stopCurrentMelody();
		tone(tonePin, 523, 100);  // C5
		anyKeyPressed = true;
		lastToneTime = millis();
	}
	if (digitalRead(5) == HIGH) {
		shouldStopMelody = false;
		tetrisTheme();
	}
	
	\end{lstlisting}
	\caption{Condicionales}
	\label{fig:codigo-led}
\end{figure}
La figura 5 completa el teclado de piano y agrega una función para tocar una melodía específica.

Cuando el pin 6 está en HIGH, el programa detiene la melodía actual y reproduce la nota C5 (Do en una octava más alta) en el pin de tono (tonePin) a 523 Hz durante 100 ms. Marca que se ha presionado una tecla (anyKeyPressed = true) y actualiza el tiempo en lastToneTime.

Si el pin 5 está en HIGH, el programa habilita la reproducción de la melodía de Tetris (shouldStopMelody = false) y llama a la función tetrisTheme() para iniciar la melodía.
 