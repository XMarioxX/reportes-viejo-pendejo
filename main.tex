\documentclass[12pt]{article}
% Paquetes básicos primero
\usepackage[utf8]{inputenc}
\usepackage[T1]{fontenc}
\usepackage{ragged2e}
\usepackage{textcomp}
\usepackage{gensymb}
\usepackage{siunitx} 
\usepackage[spanish]{babel}
\usepackage{multicol}

% Geometría y diseño
\usepackage{geometry}
\geometry{
	letterpaper,
	left=2.5cm,
	right=2.5cm,
	top=3cm,
	bottom=3cm,
	headheight=60pt,
	headsep=1cm
}

% Paquetes de formato
\usepackage{setspace}
\usepackage{graphicx}
\usepackage{fancyhdr}
\usepackage{xcolor}
\usepackage{enumitem}
\usepackage{lastpage}
\usepackage{background}
\usepackage{listings}
\usepackage{tcolorbox}
\usepackage{fontawesome}
\usepackage{titlesec}
\usepackage{float}

% Paquetes para referencias
\usepackage[round]{natbib}  % Configuración específica para citas
\bibliographystyle{plainnat}  % Estilo de bibliografía

% URLs y enlaces (hyperref debe ir al final)
\usepackage{url}
\usepackage[hidelinks]{hyperref}

% Configuración de colores personalizados
\definecolor{tecnmBlue}{RGB}{0, 43, 89}
\definecolor{tecnmWhite}{RGB}{255, 255, 255}
\definecolor{codeBackground}{RGB}{240,240,240}
\definecolor{comentarios}{RGB}{63,127,95}
\definecolor{identificador}{RGB}{127,0,85}
\definecolor{cadena}{RGB}{42,0,255}
\definecolor{arduinoBlue}{RGB}{0,129,132}

% Configuración de listings para código Arduino
\setlength{\tabcolsep}{0.5pt} % Ajusta esto fuera del estilo

\lstdefinestyle{arduinoStyle}{
	language=C++,
	backgroundcolor=\color{codeBackground},
	basicstyle=\ttfamily\scriptsize,
	columns=fixed,        % Cambiado a fixed para control exacto
	basewidth={0.5em},    % Controla el ancho base de los caracteres
	basewidth=0.4em,      % Hace los espacios más pequeños
	breaklines=true,
	captionpos=b,
	commentstyle=\color{comentarios},
	keywordstyle=\color{identificador},
	stringstyle=\color{cadena},
	numbers=left,
	numberstyle=\tiny\color{gray},
	stepnumber=1,
	numbersep=2pt,
	showspaces=false,
	showstringspaces=false,
	frame=single,
	frameround=tttt,
	framesep=1mm,
	rulecolor=\color{arduinoBlue},
	emph={pinMode,digitalWrite,HIGH,LOW,INPUT,OUTPUT,tone,noTone,delay,delayMicroseconds,millis,INPUT_PULLUP},
	emphstyle={\color{blue}\bfseries},
	morekeywords={setup,loop},
	linewidth=\columnwidth,
	xleftmargin=0.3cm,
	xrightmargin=0.3cm,
	float=h,
	floatplacement=H,
	belowskip=-0.2cm,
	aboveskip=0.2cm
}

% Configuración de cajas personalizadas
\newtcolorbox{componentBox}[2][]{
	colback=tecnmBlue!5,
	colframe=tecnmBlue!40!black,
	title=#2,
	width=\columnwidth,
	#1
}

% Configuración del encabezado y pie de página
\pagestyle{fancy}
\fancyhf{}
\renewcommand{\headrulewidth}{0pt}
\renewcommand{\footrulewidth}{0pt}

% Definir comandos para reducir la carga
\newcommand{\smallgray}[1]{{\scriptsize\textcolor{gray}{#1}}}

\fancyhead[L]{%
	\hspace*{0.5cm}\raisebox{-2.2\height}{\includegraphics[height=1cm]{images/logotecnm.png}}%
}
\fancyhead[R]{%
	\raisebox{-1.7\height}{%
		\begin{tabular}[b]{r}
			{\scriptsize\textbf{Instituto Tecnológico de Morelia}}\\[-0.6em]
			\smallgray{Subdirección Académica}\\[-0.6em]
			\smallgray{Departamento de Sistemas y Computación}\\[-0.6em]
			\smallgray{Laboratorio de Sistemas Programables}\\[-0.4em]
			{\scriptsize Página \thepage\ de \pageref{LastPage}}
		\end{tabular}%
	}%
}

\makeatletter
\renewcommand\@biblabel[1]{#1.}
\makeatother

\fancyfoot[L]{%
	\small
	Implementación de un Transistores BJT y MOSFET
	
	\vspace{0.1cm}
	
	\small\textcolor{gray}{Plantilla de Reporte Laboratorio Sistemas Programables}
}
\fancyfoot[R]{%
	\small
	\textbf{Equipo 1}%
}

% Definir comando para citas
\newcommand{\cita}[1]{\cite{#1}}

% Estilo de títulos
\titleformat{\section}
{\color{tecnmBlue}\Large\bfseries}
{\thesection}{1em}{}

\titleformat{\subsection}
{\color{tecnmBlue}\large\bfseries}
{\thesubsection}{1em}{}

% Configuración de siunitx
\sisetup{
	detect-all,
	detect-family,
	per-mode=symbol,
	range-phrase=--,
	range-units=single,
	output-decimal-marker={,}
}

\begin{document}
	\pagenumbering{arabic}
	
	% Portada (una columna)
	% portada.tex
% Configuración del fondo solo para la portada
\backgroundsetup{
	scale=1,
	angle=0,
	opacity=0.2,
	contents={\includegraphics[width=\paperwidth,height=\paperheight]{images/fondo_portada.png}}
}
\BgThispage

\begin{center}
	\vspace*{2cm}
	
	\Large\textbf{Implementación del Buffer}
	
	\vspace{1.5cm}
	
	Autor(es):
	
	\vspace{0.5cm}
	
	Mario Eduardo Sánchez Mejía\\
	Fidel Alberto Zarco Áviles
	
	\vspace{0.3cm}
	
	\small{21120721@morelia.tecnm.mx}\\
	\small{l20121258@morelia.tecnm.mx}
	
	\vspace{1.5cm}
	
	Asesor(@s):
	
	\vspace{0.5cm}
	
	Luis Ulises Chávez Campos
	
	\vspace{0.5cm}
	
	\begin{abstract}
		\noindent
		\justifying
		Se desarrolla un sistema de piano digital mediante Arduino, integrando elementos electrónicos como buzzers, LEDs y botones pulsadores. El proyecto permite la comprensión de generación de tonos musicales a través de programación, empleando un buzzer pasivo y la función tone() de Arduino. El sistema implementa las ocho notas musicales básicas y una función para reproducir melodías MIDI convertidas. La práctica integra conceptos de electrónica digital, programación y principios musicales fundamentales.
		
		\vspace{0.5cm}
		\noindent
		\textbf{Palabras clave:} Arduino, Electrónica Digital, Programación, Piano Digital, Sistemas Programables
	\end{abstract}
\end{center}

% Desactivar el fondo para las siguientes páginas
\clearpage
\backgroundsetup{contents={}}
	% Inicio de dos columnas para el contenido principal
	\begin{multicols}{2}
		\setlength{\columnsep}{1cm} % Espacio entre columnas
		\tableofcontents
		\listoffigures
		% introduccion_componentes.tex

% Configuración preliminar para ajustar el espacio
\setlength{\parindent}{0pt}
\setlength{\parskip}{6pt}

\section{Introducción}
Esta práctica se centra en la creación de un piano básico utilizando Arduino, donde aprenderemos a integrar múltiples componentes electrónicos para crear un sistema interactivo musical \citep{monk2017programming}. Los pianos electrónicos modernos utilizan circuitos y componentes para generar distintas notas musicales, y en esta práctica simularemos este funcionamiento de manera simplificada \citep{evans2011arduino}.


\subsection{LEDs (Diodos Emisores de Luz)}
\begin{componentBox}{Características y Conexión \citep{platt2014encyclopedia}}
	\begin{itemize}[leftmargin=*,itemsep=1pt,parsep=1pt]
		\item \textbf{Polaridad}:
		\begin{itemize}[itemsep=0pt,parsep=0pt]
			\item Ánodo (+): Pata más larga
			\item Cátodo (-): Pata más corta
		\end{itemize}
		\item \textbf{Especificaciones}:
		\begin{itemize}[itemsep=0pt,parsep=0pt]
			\item Voltaje típico: \SI{2.0}{\volt} - \SI{3.3}{\volt}
			\item Corriente: \SI{20}{\milli\ampere}
		\end{itemize}
		\item \textbf{Conexión}:
		\begin{itemize}[itemsep=0pt,parsep=0pt]
			\item Ánodo → Resistencia → Pin
			\item Cátodo → GND
		\end{itemize}
	\end{itemize}
\end{componentBox}

\begin{figure}[H]
	\centering
	\begin{lstlisting}[
		language=C++,
		basicstyle=\ttfamily\scriptsize,
		numbers=left,
		numberstyle=\tiny\color{gray},
		backgroundcolor=\color{codeBackground},
		commentstyle=\color{comentarios},
		keywordstyle=\color{identificador},
		stringstyle=\color{cadena},
		emph={pinMode,digitalWrite,HIGH,LOW},
		emphstyle={\color{blue}\bfseries},
		frame=single,
		frameround=tttt,
		rulecolor=\color{arduinoBlue},
		xleftmargin=3mm,
		xrightmargin=2mm,
		breaklines=false,
		columns=fullflexible,
		keepspaces=true
		basewidth=0.5em 
]
const int ledPin = 13;

void setup() {
	pinMode(ledPin, OUTPUT);
}

void loop() {
	digitalWrite(ledPin, HIGH);  // Encender
	delay(1000);
	digitalWrite(ledPin, LOW);   // Apagar
	delay(1000);
}
	\end{lstlisting}
	\caption{Código de control de LED en Arduino}
	\label{fig:codigo-led}
\end{figure}

\subsection{Botones Pulsadores}
\begin{componentBox}{Características y Conexión \citep{scherz2016practical}}
	\begin{itemize}[leftmargin=*,itemsep=1pt,parsep=1pt]
		\item \textbf{Tipo}: Interruptores (NO)
		\item \textbf{Características}:
		\begin{itemize}[itemsep=0pt,parsep=0pt]
			\item Sin polaridad específica
			\item Requieren resistencia pull-up
			\item Estado normal: Abierto
		\end{itemize}
		\item \textbf{Conexión}:
		\begin{itemize}[itemsep=0pt,parsep=0pt]
			\item Terminal 1 → Pin Arduino
			\item Terminal 2 → GND
			\item R pull-up \SI{10}{\kilo\ohm} → \SI{5}{\volt}
		\end{itemize}
	\end{itemize}
\end{componentBox}

\subsection{Buzzer Pasivo}
\begin{componentBox}{Características y Operación \citep{arduino_tone}}
	\begin{itemize}[leftmargin=*,itemsep=1pt,parsep=1pt]
		\item \textbf{Características}:
		\begin{itemize}[itemsep=0pt,parsep=0pt]
			\item Sin oscilador interno
			\item Requiere señal PWM
			\item Voltaje: \SI{3}{\volt}-\SI{12}{\volt}
		\end{itemize}
		\item \textbf{Conexión}:
		\begin{itemize}[itemsep=0pt,parsep=0pt]
			\item (+) → Pin PWM Arduino
			\item (-) → GND
		\end{itemize}
		\item \textbf{Notas musicales}:
		\begin{multicols}{2}
			\begin{itemize}[itemsep=0pt,parsep=0pt]
				\item DO (C4): \SI{261.63}{\hertz}
				\item RE (D4): \SI{293.66}{\hertz}
				\item MI (E4): \SI{329.63}{\hertz}
				\item FA (F4): \SI{349.23}{\hertz}
				\item SOL (G4): \SI{392.00}{\hertz}
				\item LA (A4): \SI{440.00}{\hertz}
				\item SI (B4): \SI{493.88}{\hertz}
				\item DO (C5): \SI{523.25}{\hertz}
			\end{itemize}
		\end{multicols}
	\end{itemize}
\end{componentBox}

\subsection{Resistencias}
\begin{componentBox}{Tipos y Usos \citep{platt2014encyclopedia}}
	\begin{itemize}[leftmargin=*,itemsep=1pt,parsep=1pt]
		\item \textbf{Para LEDs}:
		\begin{itemize}[itemsep=0pt,parsep=0pt]
			\item Rango: \SI{220}{\ohm} - \SI{1}{\kilo\ohm}
			\item Limitan corriente
			\item R = $\frac{V_{\text{fuente}} - V_{\text{led}}}{I_{\text{led}}}$
		\end{itemize}
		\item \textbf{Pull-up}:
		\begin{itemize}[itemsep=0pt,parsep=0pt]
			\item Valor: \SI{10}{\kilo\ohm}
			\item Mantiene estado lógico
		\end{itemize}
		\item \textbf{Código de colores}:
		\begin{itemize}[itemsep=0pt,parsep=0pt]
			\item \SI{220}{\ohm}: Rojo-Rojo-Marrón
			\item \SI{1}{\kilo\ohm}: Marrón-Negro-Rojo
			\item \SI{10}{\kilo\ohm}: Marrón-Negro-Naranja
		\end{itemize}
	\end{itemize}
\end{componentBox}

\section{Consideraciones de Diseño}
\begin{componentBox}{Recomendaciones Importantes \citep{banzi2009getting}}
	\begin{itemize}[leftmargin=*,itemsep=1pt,parsep=1pt]
		\item \textbf{Pines Arduino}:
		\begin{itemize}[itemsep=0pt,parsep=0pt]
			\item PWM (3,5,6,9,10,11) → buzzer
			\item LEDs en pines consecutivos
			\item Botones con pull-up interno
		\end{itemize}
		\item \textbf{Seguridad}:
		\begin{itemize}[itemsep=0pt,parsep=0pt]
			\item Verificar polaridad
			\item Máx. \SI{40}{\milli\ampere}/pin
			\item Resistencias adecuadas
			\item Desconectar al modificar
		\end{itemize}
		\item \textbf{Código}:
		\begin{itemize}[itemsep=0pt,parsep=0pt]
			\item Constantes para pines/notas
			\item Debounce en botones
			\item Funciones estructuradas
		\end{itemize}
	\end{itemize}
\end{componentBox}

\section{Objetivo del Proyecto}
\begin{componentBox}{Objetivos}
	El objetivo es crear un piano digital funcional que integre múltiples componentes electrónicos \citep{evans2011arduino}. Se desarrollarán habilidades en:
	\begin{itemize}[leftmargin=*,itemsep=1pt]
		\item Manejo de entradas/salidas digitales
		\item Generación de tonos con PWM
		\item Interacciones usuario-dispositivo
		\item Sistemas multicomponente
	\end{itemize}
\end{componentBox}
		%% desarrollo_practica.tex

\section{Desarrollo de la Practica}

\subsection{Material Utilizado}
\begin{componentBox}{Lista de Componentes}
	\begin{itemize}[leftmargin=*,itemsep=1pt,parsep=1pt]
		\item 1 Placa Arduino UNO
		\item 9 Botones pulsadores
		\item 9 LEDs
		\item 9 Resistencias de \SI{1}{\kilo\ohm} para LEDs
		\item 9 Resistencias de \SI{10}{\kilo\ohm} para pull-up
		\item 1 Buzzer pasivo
		\item Cables jumper
		\item 1 Protoboard
	\end{itemize}
\end{componentBox}

\subsection{Esquematico del Circuito}
\begin{componentBox}{Conexiones Principales}
	[Incluir imagen del esquematico]
	\begin{itemize}[leftmargin=*,itemsep=1pt,parsep=1pt]
		\item \textbf{LEDs}: Conectados a pines 11-19
		\item \textbf{Botones}: Conectados a pines 2-10
		\item \textbf{Buzzer}: Conectado al pin PWM 20
	\end{itemize}
\end{componentBox}

\subsection{Explicacion de las Conexiones}
\begin{componentBox}{Detalles de Conexion}
	\begin{itemize}[leftmargin=*,itemsep=1pt,parsep=1pt]
		\item \textbf{Conexion de LEDs}:
		\begin{itemize}[itemsep=0pt,parsep=0pt]
			\item Anodo -> Resistencia \SI{1}{\kilo\ohm} -> Pin Arduino
			\item Catodo -> GND
			\item Resistencia limita corriente a \SI{20}{\milli\ampere}
		\end{itemize}
		
		\item \textbf{Conexion de Botones}:
		\begin{itemize}[itemsep=0pt,parsep=0pt]
			\item Terminal 1 -> Pin Arduino
			\item Terminal 2 -> GND
			\item Resistencia pull-up interna habilitada
		\end{itemize}
		
		\item \textbf{Conexion del Buzzer}:
		\begin{itemize}[itemsep=0pt,parsep=0pt]
			\item Terminal positivo -> Pin PWM Arduino
			\item Terminal negativo -> GND
			\item No requiere resistencia limitadora
		\end{itemize}
	\end{itemize}
\end{componentBox}

\subsection{Descripcion del Codigo}
\begin{componentBox}{Estructura del Programa}
	\begin{itemize}[leftmargin=*,itemsep=1pt,parsep=1pt]
		\item \textbf{Variables Globales}:
		\begin{itemize}[itemsep=0pt,parsep=0pt]
			\item Arrays para pines de botones y LEDs
			\item Constantes para frecuencias de notas
			\item Pin designado para el buzzer
		\end{itemize}
		
		\item \textbf{Funcion setup()}:
		\begin{itemize}[itemsep=0pt,parsep=0pt]
			\item Configura pines de botones como INPUT\_PULLUP
			\item Configura pines de LEDs como OUTPUT
			\item Inicializa pin del buzzer
		\end{itemize}
		
		\item \textbf{Funcion loop()}:
		\begin{itemize}[itemsep=0pt,parsep=0pt]
			\item Lee estado de botones
			\item Genera tonos correspondientes
			\item Controla LEDs asociados
		\end{itemize}
		
		\item \textbf{Funcion playMelody()}:
		\begin{itemize}[itemsep=0pt,parsep=0pt]
			\item Implementa melodia especial
			\item Control de tiempos y secuencias
		\end{itemize}
	\end{itemize}
\end{componentBox}

\subsection{Observaciones y Comentarios}
\begin{componentBox}{Consideraciones Importantes}
	\begin{itemize}[leftmargin=*,itemsep=1pt,parsep=1pt]
		\item El uso de resistencias pull-up internas simplifica el circuito
		\item La funcion tone() bloquea algunas interrupciones
		\item Se recomienda implementar debounce en los botones
		\item La melodia puede personalizarse segun necesidades
		\item Los LEDs proporcionan retroalimentacion visual util
		\item El codigo es escalable para mas notas/botones
	\end{itemize}
	
	\begin{itemize}[leftmargin=*,itemsep=1pt,parsep=1pt]
		\item \textbf{Mejoras Posibles}:
		\begin{itemize}[itemsep=0pt,parsep=0pt]
			\item Implementar control de volumen
			\item Agregar mas melodias predefinidas
			\item Mejorar el manejo de multiples botones
			\item Anadir efectos de sonido adicionales
		\end{itemize}
	\end{itemize}
\end{componentBox}
		%\section{Documentacion del Piano Digital con Tema de Tetris}

\subsection{Descripcion General}
Este proyecto implementa un piano digital con Arduino que incluye ocho notas musicales basicas y una funcion especial que reproduce el tema musical de Tetris. El sistema utiliza botones como teclas de piano y proporciona retroalimentacion visual mediante un LED.

\begin{figure}[H]
	\centering
	\begin{lstlisting}[
		language=C++,
		basicstyle=\ttfamily\scriptsize,
		numbers=left,
		numberstyle=\tiny\color{gray},
		backgroundcolor=\color{codeBackground},
		commentstyle=\color{comentarios},
		keywordstyle=\color{identificador},
		stringstyle=\color{cadena},
		emph={pinMode,digitalWrite,tone,delay,HIGH,LOW},
		emphstyle={\color{blue}\bfseries},
		frame=single,
		frameround=tttt,
		rulecolor=\color{arduinoBlue},
		xleftmargin=5mm,
		xrightmargin=2mm,
		breaklines=false,
		columns=fullflexible,
		keepspaces=true,
		basewidth=0.5em 
]
int tonePin = 4;    
int ledMusic = 3;   
	\end{lstlisting}
	\caption{Definicion de Variables Globales}
	\label{fig:variables-globales}
\end{figure}

La figura \ref{fig:variables-globales} muestra la definicion de los pines principales del sistema. El \texttt{tonePin} se utiliza para la generacion de sonidos, mientras que \texttt{ledMusic} controla el LED indicador.

\subsection{Configuracion Inicial}
\begin{figure}[H]
	\centering
	\begin{lstlisting}[
		language=C++,
		basicstyle=\ttfamily\scriptsize,
		numbers=left,
		numberstyle=\tiny\color{gray},
		backgroundcolor=\color{codeBackground},
		commentstyle=\color{comentarios},
		keywordstyle=\color{identificador},
		stringstyle=\color{cadena},
		emph={pinMode,digitalWrite,tone,delay,HIGH,LOW},
		emphstyle={\color{blue}\bfseries},
		frame=single,
		frameround=tttt,
		rulecolor=\color{arduinoBlue},
		xleftmargin=5mm,
		xrightmargin=2mm,
		breaklines=false,
		columns=fullflexible,
		keepspaces=true,
		basewidth=0.5em 
]
void setup() {
	for (int pin = 5; pin <= 13; pin++) {
		pinMode(pin, INPUT);    
	}
	pinMode(tonePin, OUTPUT);   
	pinMode(ledMusic, OUTPUT);  
}
	\end{lstlisting}
	\caption{Funcion de Configuracion}
	\label{fig:setup-function}
\end{figure}

La funcion \texttt{setup()} mostrada en la figura \ref{fig:setup-function} realiza la configuracion inicial del sistema. Los pines 5-13 se configuran como entradas para los botones, el pin 4 (tonePin) como salida para el buzzer, y el pin 3 (ledMusic) como salida para el LED indicador.

\subsection{Bucle Principal y Control de Notas}
\begin{figure}[H]
	\centering
	\begin{lstlisting}[
		language=C++,
		basicstyle=\ttfamily\scriptsize,
		numbers=left,
		numberstyle=\tiny\color{gray},
		backgroundcolor=\color{codeBackground},
		commentstyle=\color{comentarios},
		keywordstyle=\color{identificador},
		stringstyle=\color{cadena},
		emph={pinMode,digitalWrite,tone,delay,HIGH,LOW},
		emphstyle={\color{blue}\bfseries},
		frame=single,
		frameround=tttt,
		rulecolor=\color{arduinoBlue},
		xleftmargin=5mm,
		xrightmargin=2mm,
		breaklines=false,
		columns=fullflexible,
		keepspaces=true,
		basewidth=0.5em 
]
void loop() {
	if (digitalRead(13) == HIGH) {
		tone(tonePin, 262, 100);  
	}
	if (digitalRead(12) == HIGH) {
		tone(tonePin, 294, 100);  
	}
	if (digitalRead(11) == HIGH) {
		tone(tonePin, 330, 100);  
	}
	if (digitalRead(10) == HIGH) {
		tone(tonePin, 349, 100);  
	}
	if (digitalRead(9) == HIGH) {
		tone(tonePin, 392, 100);  
	}
	if (digitalRead(8) == HIGH) {
		tone(tonePin, 440, 100);  
	}
	if (digitalRead(7) == HIGH) {
		tone(tonePin, 494, 100);  
	}
	if (digitalRead(6) == HIGH) {
		tone(tonePin, 523, 100);  
	}
	if (digitalRead(5) == HIGH) {
		digitalWrite(ledMusic, HIGH);
		tetrisTheme();
		digitalWrite(ledMusic, LOW);
	}
	delay(100);  
}
	\end{lstlisting}
	\caption{Bucle Principal y Control de Notas}
	\label{fig:loop-function}
\end{figure}

\subsection{Funcion Auxiliar de Control LED-Tono}
\begin{figure}[H]
	\centering
	\begin{lstlisting}[
		language=C++,
		basicstyle=\ttfamily\scriptsize,
		numbers=left,
		numberstyle=\tiny\color{gray},
		backgroundcolor=\color{codeBackground},
		commentstyle=\color{comentarios},
		keywordstyle=\color{identificador},
		stringstyle=\color{cadena},
		emph={pinMode,digitalWrite,tone,delay,HIGH,LOW},
		emphstyle={\color{blue}\bfseries},
		frame=single,
		frameround=tttt,
		rulecolor=\color{arduinoBlue},
		xleftmargin=5mm,
		xrightmargin=2mm,
		breaklines=false,
		columns=fullflexible,
		keepspaces=true,
		basewidth=0.5em 
]
void toneLed(int ledState, int frequency, 
		int duration) {
	digitalWrite(ledMusic, ledState);   
	tone(tonePin, frequency, duration); 
	delay(duration);                    
	digitalWrite(ledMusic, LOW);        
}
	\end{lstlisting}
	\caption{Funcion de Control LED-Tono}
	\label{fig:toneled-function}
\end{figure}

\subsection{Implementacion del Tema de Tetris}
\begin{figure}[H]
	\centering
	\begin{lstlisting}[
		language=C++,
		basicstyle=\ttfamily\scriptsize,
		numbers=left,
		numberstyle=\tiny\color{gray},
		backgroundcolor=\color{codeBackground},
		commentstyle=\color{comentarios},
		keywordstyle=\color{identificador},
		stringstyle=\color{cadena},
		emph={pinMode,digitalWrite,tone,delay,HIGH,LOW},
		emphstyle={\color{blue}\bfseries},
		frame=single,
		frameround=tttt,
		rulecolor=\color{arduinoBlue},
		xleftmargin=5mm,
		xrightmargin=2mm,
		breaklines=false,
		columns=fullflexible,
		keepspaces=true,
		basewidth=0.5em 
]
void tetrisTheme() {
	toneLed(HIGH, 659, 250);  
	toneLed(HIGH, 494, 125);  
	toneLed(HIGH, 523, 125);  
	toneLed(HIGH, 587, 250);  
	toneLed(HIGH, 523, 125);  
	toneLed(HIGH, 494, 125);  
	toneLed(HIGH, 440, 250);  
	toneLed(HIGH, 440, 125);  
	toneLed(HIGH, 523, 125);  
	toneLed(HIGH, 659, 250);  
	toneLed(HIGH, 587, 125);  
	toneLed(HIGH, 523, 125);  
	toneLed(HIGH, 494, 375);  
	toneLed(HIGH, 523, 125);  
	toneLed(HIGH, 587, 250);  
	toneLed(HIGH, 659, 250);  
	toneLed(HIGH, 523, 250);  
	toneLed(HIGH, 440, 250);  
	toneLed(HIGH, 440, 500);  
	
	delay(250);  
	
	toneLed(HIGH, 494, 250);  
	toneLed(HIGH, 587, 250);  
	toneLed(HIGH, 659, 250);  
	toneLed(HIGH, 698, 250);  
	toneLed(HIGH, 659, 125);  
	toneLed(HIGH, 587, 125);  
	toneLed(HIGH, 523, 375);  
	toneLed(HIGH, 523, 125);  
	toneLed(HIGH, 587, 250);  
	toneLed(HIGH, 659, 250);  
	toneLed(HIGH, 523, 250);  
	toneLed(HIGH, 494, 250);  
	toneLed(HIGH, 440, 500);  
	
	delay(500);  
}
	\end{lstlisting}
	\caption{Implementacion del Tema de Tetris}
	\label{fig:tetris-theme}
\end{figure}

\subsection{Tabla de Frecuencias}
\begin{table}*[H]
	\centering
	\begin{tabular}{|l|c|l|}
		\hline
		\textbf{Nota} & \textbf{Frecuencia (Hz)} & \textbf{Pin de Control} \\
		\hline
		Do (C4) & 262 & 13 \\
		Re (D4) & 294 & 12 \\
		Mi (E4) & 330 & 11 \\
		Fa (F4) & 349 & 10 \\
		Sol (G4) & 392 & 9 \\
		La (A4) & 440 & 8 \\
		Si (B4) & 494 & 7 \\
		Do (C5) & 523 & 6 \\
		Tetris Theme & Variable & 5 \\
		\hline
	\end{tabular}
	\caption{Asignacion de Notas y Pines de Control}
	\label{tab:frequency-pins}
\end{table}

\subsection{Control de Tiempo y Sincronizacion}
\begin{figure}[H]
	\centering
	\begin{lstlisting}[
		language=C++,
		basicstyle=\ttfamily\scriptsize,
		numbers=left,
		numberstyle=\tiny\color{gray},
		backgroundcolor=\color{codeBackground},
		commentstyle=\color{comentarios},
		keywordstyle=\color{identificador},
		stringstyle=\color{cadena},
		emph={delay,tone},
		emphstyle={\color{blue}\bfseries},
		frame=single,
		frameround=tttt,
		rulecolor=\color{arduinoBlue},
		xleftmargin=5mm,
		xrightmargin=2mm,
		breaklines=false,
		columns=fullflexible,
		keepspaces=true,
		basewidth=0.5em 
]
delay(100);    
delay(250);    
delay(500);    

tone(tonePin, frequency, duration);  
	\end{lstlisting}
	\caption{Mecanismos de Control Temporal}
	\label{fig:timing-control}
\end{figure}

\subsubsection{Estructura del Tema Musical}
\begin{figure}[H]
	\centering
	\begin{lstlisting}[
		language=C++,
		basicstyle=\ttfamily\scriptsize,
		numbers=left,
		numberstyle=\tiny\color{gray},
		backgroundcolor=\color{codeBackground},
		commentstyle=\color{comentarios},
		keywordstyle=\color{identificador},
		stringstyle=\color{cadena},
		emph={toneLed,delay,HIGH},
		emphstyle={\color{blue}\bfseries},
		frame=single,
		frameround=tttt,
		rulecolor=\color{arduinoBlue},
		xleftmargin=5mm,
		xrightmargin=2mm,
		breaklines=false,
		columns=fullflexible,
		keepspaces=true,
		basewidth=0.5em 
]
toneLed(HIGH, 659, 250);  
toneLed(HIGH, 494, 125);  
toneLed(HIGH, 523, 125);  

toneLed(HIGH, 587, 250);  
toneLed(HIGH, 523, 125);  
toneLed(HIGH, 494, 125);  

toneLed(HIGH, 494, 250);  
toneLed(HIGH, 587, 250);  
toneLed(HIGH, 659, 250);  
	\end{lstlisting}
	\caption{Estructura Musical del Tema de Tetris}
	\label{fig:music-structure}
\end{figure}

\subsection{Diagrama de Conexiones}
\begin{figure}[H]
	\centering
	\begin{lstlisting}[
		language=C++,
		basicstyle=\ttfamily\scriptsize,
		numbers=left,
		numberstyle=\tiny\color{gray},
		backgroundcolor=\color{codeBackground},
		frame=single,
		frameround=tttt,
		rulecolor=\color{arduinoBlue},
		xleftmargin=5mm,
		xrightmargin=2mm,
		breaklines=false,
		columns=fullflexible,
		keepspaces=true,
		basewidth=0.5em 
]
Arduino    |    Componente
-----------------------
Pin 4      ->   Buzzer (+)
GND        ->   Buzzer (-)
Pin 3      ->   LED Musical (+)
GND        ->   LED Musical (-)
Pin 13     ->   Boton Do (C4)
Pin 12     ->   Boton Re (D4)
Pin 11     ->   Boton Mi (E4)
Pin 10     ->   Boton Fa (F4)
Pin 9      ->   Boton Sol (G4)
Pin 8      ->   Boton La (A4)
Pin 7      ->   Boton Si (B4)
Pin 6      ->   Boton Do (C5)
Pin 5      ->   Boton Tetris
	\end{lstlisting}
	\caption{Diagrama de Conexiones del Sistema}
	\label{fig:connection-diagram}
\end{figure}

\section{Detalles de Implementación}

\subsection{Estructura de Datos}
La implementación utiliza estructuras de datos simples y eficientes para el manejo de las notas musicales. El siguiente código muestra la organización básica:

\begin{figure}[H]
\centering
\begin{lstlisting}[
	language=C++,
	basicstyle=\ttfamily\scriptsize,
	numbers=left,
	numberstyle=\tiny\color{gray},
	backgroundcolor=\color{codeBackground},
	frame=single,
	frameround=tttt,
	rulecolor=\color{arduinoBlue},
	xleftmargin=5mm,
	xrightmargin=2mm,
	breaklines=false,
	columns=fullflexible,
	keepspaces=true,
	basewidth=0.5em 
]
const int NOTAS_BASICAS = 8;
const int FRECUENCIAS[NOTAS_BASICAS] = {262, 294, 330, 349, 392, 440, 494, 523};
const int PINES_BOTONES[NOTAS_BASICAS] = {13, 12, 11, 10, 9, 8, 7, 6};
\end{lstlisting}
\caption{Definición de Arrays para Notas y Pines}
\label{fig:arrays-definition}
\end{figure}

\subsection{Control de Tiempo Avanzado}
El sistema implementa un control de tiempo preciso para la reproducción musical. A continuación se muestra la implementación detallada:

\begin{figure}[H]
\centering
\begin{lstlisting}[
	language=C++,
	basicstyle=\ttfamily\scriptsize,
	numbers=left,
	numberstyle=\tiny\color{gray},
	backgroundcolor=\color{codeBackground},
	frame=single,
	frameround=tttt,
	rulecolor=\color{arduinoBlue},
	xleftmargin=5mm,
	xrightmargin=2mm
]
unsigned long previousMillis = 0;
const long interval = 100;

void timedTone(int frequency, int duration) {
unsigned long currentMillis = millis();
if (currentMillis - previousMillis >= interval) {
	previousMillis = currentMillis;
	tone(tonePin, frequency, duration);
}
}
\end{lstlisting}
\caption{Implementación de Control de Tiempo}
\label{fig:time-control}
\end{figure}

\subsection{Sistema de Efectos LED}
El sistema de iluminación LED se ha implementado con diferentes patrones para indicar el estado del piano:

\begin{figure}[H]
\centering
\begin{lstlisting}[
	language=C++,
	basicstyle=\ttfamily\scriptsize,
	numbers=left,
	numberstyle=\tiny\color{gray},
	backgroundcolor=\color{codeBackground},
	frame=single,
	frameround=tttt,
	rulecolor=\color{arduinoBlue},
	xleftmargin=5mm,
	xrightmargin=2mm
]
void ledPattern(int pattern) {
	switch(pattern) {
		case 0:  
		digitalWrite(ledMusic, HIGH);
		delay(50);
		digitalWrite(ledMusic, LOW);
		break;
		case 1:  
		for(int i = 0; i < 3; i++) {
			digitalWrite(ledMusic, HIGH);
			delay(100);
			digitalWrite(ledMusic, LOW);
			delay(100);
		}
		break;
	}
}
\end{lstlisting}
\caption{Sistema de Patrones LED}
\label{fig:led-patterns}
\end{figure}

\subsection{Optimización de Memoria}
La siguiente implementación muestra cómo se ha optimizado el uso de memoria:

\begin{figure}[H]
\centering
\begin{lstlisting}[
	language=C++,
	basicstyle=\ttfamily\scriptsize,
	numbers=left,
	numberstyle=\tiny\color{gray},
	backgroundcolor=\color{codeBackground},
	frame=single,
	frameround=tttt,
	rulecolor=\color{arduinoBlue},
	xleftmargin=5mm,
	xrightmargin=2mm
]
struct NotaMusical {
	uint16_t frecuencia;
	uint8_t duracion;
};

const PROGMEM NotaMusical tetrisNotas[] = {
	{659, 250}, {494, 125}, {523, 125},
	{587, 250}, {523, 125}, {494, 125},
	{440, 250}, {440, 125}, {523, 125}
};
\end{lstlisting}
\caption{Estructuras de Datos Optimizadas}
\label{fig:optimized-structures}
\end{figure}

\section{Extensiones del Sistema}

\subsection{Modo de Práctica}
Se implementó un modo de práctica con el siguiente código:

\begin{figure}[H]
\centering
\begin{lstlisting}[
	language=C++,
	basicstyle=\ttfamily\scriptsize,
	numbers=left,
	numberstyle=\tiny\color{gray},
	backgroundcolor=\color{codeBackground},
	frame=single,
	frameround=tttt,
	rulecolor=\color{arduinoBlue},
	xleftmargin=5mm,
	xrightmargin=2mm
]
void modoPractica() {
static uint8_t notaActual = 0;
static unsigned long tiempoInicio = 0;

if (millis() - tiempoInicio > 2000) {
	tiempoInicio = millis();
	toneLed(HIGH, FRECUENCIAS[notaActual], 500);
	notaActual = (notaActual + 1) % NOTAS_BASICAS;
}
}
\end{lstlisting}
\caption{Implementación del Modo Práctica}
\label{fig:practice-mode}
\end{figure}

\subsection{Sistema de Detección de Errores}
Se implementó un sistema básico de detección de errores:

\begin{figure}[H]
\centering
\begin{lstlisting}[
	language=C++,
	basicstyle=\ttfamily\scriptsize,
	numbers=left,
	numberstyle=\tiny\color{gray},
	backgroundcolor=\color{codeBackground},
	frame=single,
	frameround=tttt,
	rulecolor=\color{arduinoBlue},
	xleftmargin=5mm,
	xrightmargin=2mm
]
bool verificarSistema() {
	bool sistemaCorrecto = true;
	
	for (int i = 0; i < NOTAS_BASICAS; i++) {
		pinMode(PINES_BOTONES[i], INPUT);
		if (digitalRead(PINES_BOTONES[i]) == HIGH) {
			sistemaCorrecto = false;
			ledPattern(1);
		}
	}
	
	return sistemaCorrecto;
}
\end{lstlisting}
\caption{Sistema de Verificación}
\label{fig:error-detection}
\end{figure}

\section{Conclusiones y Recomendaciones}

\subsection{Resumen de Características}
El piano digital implementado ofrece las siguientes características principales:
\begin{itemize}
\item 8 notas musicales básicas
\item Tema musical de Tetris pregrabado
\item Retroalimentación visual mediante LED
\item Sistema de control de tiempo preciso
\item Modo de práctica
\item Sistema de detección de errores
\end{itemize}

\subsection{Recomendaciones de Uso}
Para un funcionamiento óptimo del sistema, se recomienda:
\begin{itemize}
\item Verificar las conexiones antes de cada uso
\item Mantener un tiempo mínimo entre pulsaciones de botones
\item Utilizar una fuente de alimentación estable
\item Realizar pruebas periódicas del sistema de verificación
\end{itemize}
	\end{multicols}
	% Referencias (una columna)
	\clearpage
	\addcontentsline{toc}{section}{Referencias}
	\bibliography{referencias}
	
\end{document}